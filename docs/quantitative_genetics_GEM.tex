% Options for packages loaded elsewhere
\PassOptionsToPackage{unicode}{hyperref}
\PassOptionsToPackage{hyphens}{url}
%
\documentclass[
]{book}
\usepackage{amsmath,amssymb}
\usepackage{lmodern}
\usepackage{ifxetex,ifluatex}
\ifnum 0\ifxetex 1\fi\ifluatex 1\fi=0 % if pdftex
  \usepackage[T1]{fontenc}
  \usepackage[utf8]{inputenc}
  \usepackage{textcomp} % provide euro and other symbols
\else % if luatex or xetex
  \usepackage{unicode-math}
  \defaultfontfeatures{Scale=MatchLowercase}
  \defaultfontfeatures[\rmfamily]{Ligatures=TeX,Scale=1}
\fi
% Use upquote if available, for straight quotes in verbatim environments
\IfFileExists{upquote.sty}{\usepackage{upquote}}{}
\IfFileExists{microtype.sty}{% use microtype if available
  \usepackage[]{microtype}
  \UseMicrotypeSet[protrusion]{basicmath} % disable protrusion for tt fonts
}{}
\makeatletter
\@ifundefined{KOMAClassName}{% if non-KOMA class
  \IfFileExists{parskip.sty}{%
    \usepackage{parskip}
  }{% else
    \setlength{\parindent}{0pt}
    \setlength{\parskip}{6pt plus 2pt minus 1pt}}
}{% if KOMA class
  \KOMAoptions{parskip=half}}
\makeatother
\usepackage{xcolor}
\IfFileExists{xurl.sty}{\usepackage{xurl}}{} % add URL line breaks if available
\IfFileExists{bookmark.sty}{\usepackage{bookmark}}{\usepackage{hyperref}}
\hypersetup{
  pdftitle={Quantitative Genetics Graduate Education Module - Spring 2020},
  pdfauthor={William A. Cresko},
  hidelinks,
  pdfcreator={LaTeX via pandoc}}
\urlstyle{same} % disable monospaced font for URLs
\usepackage{longtable,booktabs,array}
\usepackage{calc} % for calculating minipage widths
% Correct order of tables after \paragraph or \subparagraph
\usepackage{etoolbox}
\makeatletter
\patchcmd\longtable{\par}{\if@noskipsec\mbox{}\fi\par}{}{}
\makeatother
% Allow footnotes in longtable head/foot
\IfFileExists{footnotehyper.sty}{\usepackage{footnotehyper}}{\usepackage{footnote}}
\makesavenoteenv{longtable}
\usepackage{graphicx}
\makeatletter
\def\maxwidth{\ifdim\Gin@nat@width>\linewidth\linewidth\else\Gin@nat@width\fi}
\def\maxheight{\ifdim\Gin@nat@height>\textheight\textheight\else\Gin@nat@height\fi}
\makeatother
% Scale images if necessary, so that they will not overflow the page
% margins by default, and it is still possible to overwrite the defaults
% using explicit options in \includegraphics[width, height, ...]{}
\setkeys{Gin}{width=\maxwidth,height=\maxheight,keepaspectratio}
% Set default figure placement to htbp
\makeatletter
\def\fps@figure{htbp}
\makeatother
\setlength{\emergencystretch}{3em} % prevent overfull lines
\providecommand{\tightlist}{%
  \setlength{\itemsep}{0pt}\setlength{\parskip}{0pt}}
\setcounter{secnumdepth}{5}
\usepackage{booktabs}
\usepackage{amsthm}
\makeatletter
\def\thm@space@setup{%
  \thm@preskip=8pt plus 2pt minus 4pt
  \thm@postskip=\thm@preskip
}
\makeatother
\ifluatex
  \usepackage{selnolig}  % disable illegal ligatures
\fi
\usepackage[]{natbib}
\bibliographystyle{apalike}

\title{Quantitative Genetics Graduate Education Module - Spring 2020}
\author{William A. Cresko}
\date{2021-03-31}

\begin{document}
\maketitle

{
\setcounter{tocdepth}{1}
\tableofcontents
}
\hypertarget{course-overview}{%
\chapter{Course Overview}\label{course-overview}}

This is the complete set of \emph{course materials} for the \emph{Quantitative Genetics Graduate Education Module (GEM)} at the University of Oregon for the Spring of 2020. It is written in \textbf{Markdown} so that it can be easily updated.

In book you will find nearly all the information you will need to complete the course.

\hypertarget{introduction-to-the-course}{%
\chapter{Introduction to the course}\label{introduction-to-the-course}}

This is the main set of \emph{course materials} for the \emph{Quantitative Genetics Graduate Education Module (GEM)} at the University of Oregon for the Spring of 2020. It is written in \textbf{Markdown} so that it can be easily updated.

In this book you will find nearly all the information you will need to complete the course.

\hypertarget{instructor}{%
\section{Instructor}\label{instructor}}

Dr.~Bill Cresko, \href{mailto:wcresko@uoregon.edu}{\nolinkurl{wcresko@uoregon.edu}}

\hypertarget{course-information}{%
\section{Course Information}\label{course-information}}

Virtual Class Hours: M-W-F 11 to 12 (Zoom)\\
Virtual Office Hours: F 2 to 3 or by appointment (Zoom)\\
\url{https://uoregon.zoom.us/j/99789831102}

\hypertarget{software}{%
\section{Software}\label{software}}

\begin{itemize}
\item
  Latest version of R
\item
  Latest version of RStudio
\end{itemize}

\hypertarget{inclusion-and-accessibility}{%
\section{Inclusion and Accessibility}\label{inclusion-and-accessibility}}

Please tell me your preferred pronouns and/or name, especially if it differs from the class roster. I take seriously my responsibility to create inclusive learning environments. Please notify me if there are aspects of the instruction or design of this course that result in barriers to your participation! You are also encouraged to contact the Accessible Education Center in 164 Oregon Hall at 541-346-1155 or \href{mailto:uoaec@uoregon.edu}{\nolinkurl{uoaec@uoregon.edu}}.

I am committed to making this course an inclusive and respectful learning space. Being respectful includes using preferred pronouns for your classmates. Your classmates come from a diverse set of backgrounds and experiences; please avoid assumptions or stereotypes, and aim for inclusivity. Let us know if there are classroom dynamics that impede your (or someone else's) full engagement.

Because of the COVID-19 pandemic, this course is being delivered entirely remotely. I realize that this situation makes it difficult for some students to interact with the material, for a variety of reasons. I am committed to flexibility during this stressful time and emphasize that I will work with students to overcome difficult barriers as they arise.

\hypertarget{course-schedule}{%
\chapter{Course Schedule}\label{course-schedule}}

\hypertarget{background-material}{%
\section{Background Material}\label{background-material}}

\begin{itemize}
\tightlist
\item
  Introduction to quantitative genetics
\item
  Reading materials
\item
  Introduction to R and RMarkdown
\end{itemize}

\hypertarget{heritability}{%
\section{Heritability}\label{heritability}}

\begin{itemize}
\tightlist
\item
  Genetic variation
\item
  Parent-offspring regression
\item
  Line means analysis
\item
  Environmental variation and trait distributions
\item
  Covariation among traits
\item
  The G-matrix
\end{itemize}

\hypertarget{selection-analysis}{%
\section{Selection Analysis}\label{selection-analysis}}

\begin{itemize}
\tightlist
\item
  Calculating fitness
\item
  Selection analysis
\item
  Correlation among traits
\item
  Normalization of traits
\item
  Calculation of selection gradients
\item
  Comparison of the pattern of selection across populations
\end{itemize}

\hypertarget{quantitative-genetic-mapping}{%
\section{Quantitative Genetic Mapping}\label{quantitative-genetic-mapping}}

\begin{itemize}
\tightlist
\item
  Basic phenotyping and mapping information
\item
  Quantitative Trait Loci (QTL) mapping
\item
  Epistasis
\item
  Genome Wide Association Studies (GWAS)
\end{itemize}

\hypertarget{phenotypic-traits-and-values}{%
\chapter{Phenotypic traits and values}\label{phenotypic-traits-and-values}}

\hypertarget{introduction}{%
\section{Introduction}\label{introduction}}

Many phenotypes of organisms are complex, meaning that they are the product of multiple loci interacting with one another and the environment to create smooth distributions.

\hypertarget{phenotypic-variance-and-heritability}{%
\chapter{Phenotypic variance and heritability}\label{phenotypic-variance-and-heritability}}

\hypertarget{introduction-1}{%
\section{Introduction}\label{introduction-1}}

  \bibliography{book.bib,packages.bib}

\end{document}
